% meeting_notes_1218_faithful.tex
\documentclass[11pt]{article}

\usepackage[margin=1in]{geometry}
\usepackage{setspace}
\usepackage{amsmath, amssymb}
\usepackage{booktabs}
\usepackage{enumitem}
\usepackage{hyperref}

\setstretch{1.1}
\setlength{\parskip}{0.4em}
\setlength{\parindent}{0pt}

\title{Informal Bids $\rightarrow$ Formal Round: Threshold/Interval Framework (Meeting Notes)}
\author{Austin Li}
\date{\today}

\begin{document}
\maketitle

\section{Scope and context}
\textbf{Meeting date:} 18 Dec (per handwritten notes). \\
\textbf{Document date:} \today.

This document records the threshold/interval framework discussed for modeling admission from the informal-bid stage to the formal round in an M\&A-style auction process. The key observable outcome is whether each bidder is \emph{admitted} or \emph{rejected} after submitting an informal bid. The central modeling device is a (possibly bidder-adjusted) \emph{deterministic screening threshold} that is latent to the econometrician and is characterized by inequality/interval restrictions implied by observed bids and admission decisions. Likelihood-based estimation and MCMC-style data augmentation were discussed as implementation routes.

\paragraph{Data context.}
A hand-collected sample of approximately 20 auctions was mentioned in the discussion (with a reference to Matvos and Seru, 2014, \emph{RFS}, ``conglomerates''). This dataset has not been shared with the author as of the date of this note due to the holiday season.

\section{Process timeline and information structure}
The discussion treats the auction process as proceeding along the following timeline:
\begin{enumerate}[leftmargin=2em]
    \item \textbf{Start.}
    \item \textbf{NDA / limited information access.} Bidders sign an NDA and obtain access to limited target information.
    \item \textbf{Informal bids.} Each bidder $j$ submits an informal bid $b^I_{ij}$ in auction $i$.
    \item \textbf{Admission decision (A/R).} The target admits a subset of bidders to the formal round and rejects the remainder.
    \item \textbf{Formal bids.} Admitted bidders submit formal bids (and potentially other terms).
    \item \textbf{Deal announcement.}
\end{enumerate}

\section{Notation}
Indexing:
\begin{itemize}[leftmargin=2em]
    \item $i$ indexes an \textbf{auction/deal process}.
    \item $j$ indexes a \textbf{bidder}.
\end{itemize}

Observed objects:
\begin{itemize}[leftmargin=2em]
    \item $b_{ij}^{I}$: bidder $j$'s \textbf{informal bid} in auction $i$.
    \item $X_i$: \textbf{auction/target characteristics}. The discussion noted that $X_i$ may include \emph{moments of the informal-bid vector} within an auction (e.g., mean, variance, maximum), in addition to target/auction states observed outside the bidding data.
    \item $y_{ij}$: \textbf{bidder-side characteristics} (e.g., bidder type indicators, financing certainty proxies, reputation, ``seriousness'' proxies such as number of interactions/bids and bid dispersion).
\end{itemize}

Admission sets:
\begin{itemize}[leftmargin=2em]
    \item $\mathcal{A}_i$: set of bidders \textbf{admitted} to the formal round in auction $i$.
    \item $\mathcal{R}_i$: set of bidders \textbf{rejected} after the informal round, with $\mathcal{R}_i = \{1,\dots,J\}\setminus \mathcal{A}_i$ when $J$ bidders participate.
\end{itemize}

Latent objects:
\begin{itemize}[leftmargin=2em]
    \item $b_{i}^{I*}$: an \textbf{auction-level screening threshold/cutoff}.
    \item Optionally, $b_{ij}^{I*}$: a \textbf{bidder-adjusted threshold} in auction $i$.
\end{itemize}

\section{Case (1): simplest threshold screening (auction-level cutoff)}
\subsection{Deterministic admission rule}
A baseline deterministic admission rule discussed is:
\[
j\in\mathcal{A}_i \quad \Longleftrightarrow \quad b_{ij}^{I} \ge b_{i}^{I*}.
\]
The threshold $b_i^{I*}$ is not observed by the econometrician.

\subsection{Interval restriction from admits/rejects}
If both admitted and rejected bidders are observed in auction $i$, then the data imply an interval restriction on $b_i^{I*}$:
\begin{align}
L_i &:= \max_{j \in \mathcal{R}_i} b_{ij}^{I}, \\
U_i &:= \min_{j \in \mathcal{A}_i} b_{ij}^{I},
\end{align}
so that
\[
b_{i}^{I*} \in [L_i, U_i].
\]
Toy example from the discussion: informal bids $\{18,20,22\}$ with admitted bidders $\{20,22\}$ imply $L_i=18$, $U_i=20$, hence $b_i^{I*}\in[18,20]$.

\paragraph{One-sided cases.}
\begin{itemize}[leftmargin=2em]
    \item If $\mathcal{R}_i=\emptyset$ (everyone admitted), then only an upper bound is available: $b_i^{I*}\le U_i$.
    \item If $\mathcal{A}_i=\emptyset$ (no one admitted), then only a lower bound is available: $b_i^{I*}\ge L_i$.
\end{itemize}

\section{Contrast: probit/logit versus threshold/interval}
A probabilistic admission model (e.g., probit/logit) would specify an index with an idiosyncratic error, such as
\[
\mathbf{1}\{j\in \mathcal{A}_i\} = \mathbf{1}\{ \text{index}(X_i, y_{ij}) + \varepsilon_{ij} \ge 0\},
\]
and estimate admission probabilities directly.

The threshold/interval approach instead treats admission as driven by a deterministic screening rule. Randomness in observed admission outcomes (from the econometrician's perspective) arises from unobserved components of the screening rule and/or latent objects that are not observed in the data.

\section{Parametric representation of the latent cutoff and estimation}
\subsection{Auction-level latent cutoff}
A parametric representation discussed is:
\[
b_i^{I*} = \beta^\top X_i + \nu_i^{I},
\]
where $\nu_i^{I}$ collects unobserved screening components. Combining this with the interval restriction yields:
\[
\nu_i^{I} \in [L_i - \beta^\top X_i,\; U_i - \beta^\top X_i]
\qquad (\text{for auctions with both admits and rejects}).
\]

\subsection{Likelihood-based estimation (one convenient specification)}
A common starting point is to impose a distribution on $\nu_i^{I}$ (e.g., $\nu_i^{I}\sim \mathcal{N}(0,\sigma^2)$). Under that choice, the likelihood contribution for an auction with two-sided bounds is:
\[
\Pr(L_i \le b_i^{I*} \le U_i \mid X_i)
=
\Phi\!\left(\frac{U_i-\beta^\top X_i}{\sigma}\right)
-
\Phi\!\left(\frac{L_i-\beta^\top X_i}{\sigma}\right),
\]
with one-sided analogues when only an upper or lower bound is available. Other parametric families can be substituted without changing the interval logic.

\subsection{MCMC/data-augmentation approach (outline recorded from the discussion)}
A data-augmentation approach was discussed for settings with small samples and/or extensions beyond the simplest likelihood. The recorded outline is:
\begin{enumerate}[leftmargin=2em]
    \item Initialize parameter values (e.g., $\beta_0$).
    \item For each auction, simulate latent unobservables (e.g., $\nu_i^{I}$) subject to satisfying the observed inequality restrictions implied by admits/rejects, and compute the corresponding latent thresholds $b_{i}^{I*}$.
    \item Update/estimate $\beta$ given the simulated thresholds $\{b_i^{I*}\}$ (and observed $X_i$).
    \item Draw new parameter values (or propose/accept updates) and iterate.
\end{enumerate}
This outline is compatible with standard Gibbs or Metropolis--Hastings implementations once specific distributional assumptions and conditional sampling steps are fixed.

\section{Case (2): highest informal bid may fail to advance (bidder-adjusted screening)}
The discussion emphasized that the highest informal bid need not be admitted to the formal round. A convenient way to represent this is bidder-adjusted screening:
\[
j\in\mathcal{A}_i \quad \Longleftrightarrow \quad b_{ij}^{I} \ge b_{ij}^{I*},
\qquad
b_{ij}^{I*} = b_i^{I*} + \delta^\top y_{ij} + \rho_{ij}^{I},
\]
where $y_{ij}$ includes bidder-specific attributes that can shift the effective screening bar, and $\rho_{ij}^{I}$ collects unobserved bidder-auction components.

An equivalent representation (when $\rho_{ij}^{I}$ is omitted or structured) is to define a bidder-adjusted bid:
\[
\widetilde b_{ij}^{I}(\delta) := b_{ij}^{I} - \delta^\top y_{ij},
\]
and use an auction-level threshold:
\[
j\in\mathcal{A}_i \Longleftrightarrow \widetilde b_{ij}^{I}(\delta) \ge b_i^{I*}.
\]
Then the implied interval restriction becomes:
\[
b_i^{I*} \in 
\Big[
\max_{j\in\mathcal{R}_i}\widetilde b_{ij}^{I}(\delta),
\;
\min_{j\in\mathcal{A}_i}\widetilde b_{ij}^{I}(\delta)
\Big].
\]

\paragraph{Bidder types.}
The discussion used two bidder types labelled $S$ and $F$ and noted that bidder ``seriousness'' and related attributes (e.g., number of interactions/bids, dispersion of bids) may enter $y_{ij}$. One possible interpretation of $S/F$ is strategic versus financial bidders, but the formal development uses the labels $S$ and $F$ generically.

\section{Case (3): screening depends on the target's inference about bidder valuations}
A further extension discussed is that the target's admission decision can depend on its inference about bidders' valuations, and in particular on \emph{moments of the bidder valuation distribution} (or moments of bids as proxies).

One way to represent this is to let the cutoff depend on target characteristics that include valuation moments:
\[
b_i^{I*} = \beta^\top X_i + \nu_i^{I},
\qquad
X_i = \big(\text{target/auction states},\ \text{moments of bidder valuations or bids}\big).
\]
The notes also recorded a reduced-form representation for valuations as a function of auction/target state variables (schematically),
\[
v_{ij}^{I} = \delta^\top X_i + \varepsilon_{ij},
\]
with the understanding that the target can form (estimated) moments of $\{v_{ij}^{I}\}_j$ (or of $\{b^I_{ij}\}_j$) given $X_i$ and observed bidding objects.

\paragraph{MCMC outline recorded for this case.}
The recorded algorithmic sketch is:
\begin{enumerate}[leftmargin=2em]
    \item Initialize parameter values (e.g., $\delta_0$, $\beta_0$).
    \item Simulate latent components (e.g., $\nu_i^{I}$ and bidder-level shocks) for each auction.
    \item Given current $\delta$, compute implied moments of bidder valuations (or bid-based proxies) within each auction.
    \item Update/estimate $\beta$ and $\delta$ using the computed valuation moments as controls.
    \item Draw new $(\beta,\delta)$ and iterate.
\end{enumerate}

\section{Simulation tasks (feasibility and sensitivity checks)}
The simulation tasks discussed were intended to validate whether the threshold/interval framework can recover cutoffs (and, in extensions, type-specific cutoffs) under limited observability. The numeric values below are illustrative baseline values recorded from the handwritten notes; the tasks explicitly require sensitivity analysis around these values.

\subsection{Task A (Case 1 simulation): single constant cutoff}
\textbf{Goal.} Estimate a constant cutoff $b^{I*}$ when only admission/rejection decisions are observed (together with the simulated bids/valuations, depending on the simulation design).

\textbf{Baseline DGP (illustrative).}
\begin{enumerate}[leftmargin=2em]
    \item Simulate $N$ auctions. Each auction has $J=3$ bidders.
    \item Simulate bidder valuations at the informal stage:
    \[
    v_{ij} = 1.3 + \epsilon_{ij},
    \qquad
    \epsilon_{ij}\sim \mathcal{N}(0,0.2^2),
    \qquad
    b_{ij}^{I}=v_{ij}.
    \]
    \item Set a constant admission cutoff (illustrative baseline):
    \[
    b^{I*} = 1.4.
    \]
    \item Apply admission:
    \[
    j\in\mathcal{A}_i \Longleftrightarrow b_{ij}^{I}\ge b^{I*}.
    \]
    \item Some simulated auctions will be ``incomplete'' (all admit or all reject). The notes recorded dropping incomplete auctions in a first pass.
\end{enumerate}

\textbf{Sensitivity requirements.}
Vary the illustrative numbers and assess how recovery changes:
\begin{enumerate}[leftmargin=2em]
    \item Vary $b^{I*}$ over a grid (e.g., percentiles of the simulated bid distribution).
    \item Vary the dispersion of valuations (e.g., change $0.2$).
    \item Vary the number of auctions $N$, including a small-sample stress test around $N\approx 20$.
    \item Track the frequency of incomplete auctions and quantify the selection effect of dropping them.
\end{enumerate}

\subsection{Task B (Case 2 simulation): two types, two cutoffs}
\textbf{Goal.} Estimate type-specific cutoffs when bidders fall into two types labelled $S$ and $F$.

\textbf{Baseline DGP (illustrative).}
\begin{enumerate}[leftmargin=2em]
    \item Simulate $N$ auctions. Each auction has $J=3$ bidders.
    \item Assign bidder types $T_{ij}\in\{S,F\}$ (either fixed shares or bidder-specific rules).
    \item Simulate valuations/bids as in Task A (or allow type-dependent distributions as an extension).
    \item Set type-specific admission cutoffs (illustrative baseline recorded in the notes):
    \[
    b^{I*}_S = 1.45,
    \qquad
    b^{I*}_F = 1.35.
    \]
    \item Apply admission:
    \[
    j\in\mathcal{A}_i \Longleftrightarrow b_{ij}^{I}\ge b^{I*}_{T_{ij}}.
    \]
    \item As in Task A, record and (initially) drop incomplete auctions.
\end{enumerate}

\textbf{Sensitivity requirements.}
\begin{enumerate}[leftmargin=2em]
    \item Vary the levels of $(b^{I*}_S,b^{I*}_F)$ and their gap.
    \item Vary the type share to assess identification for minority types.
    \item Compare cases where $T_{ij}$ is observed versus latent (must be inferred) and document identification issues in the latent-type case.
\end{enumerate}

\subsection{Deliverables}
\begin{enumerate}[leftmargin=2em]
    \item A short write-up (1--2 pages) reporting recovery accuracy for cutoffs/parameters, frequency of informative intervals, and sensitivity to cutoff magnitudes and type separations.
    \item Figures/tables showing estimator bias/MSE versus cutoff levels, cutoff gaps, sample size $N$, and type share.
    \item A recommendation on whether the empirical implementation should begin with a single cutoff, observed-type cutoffs, or a latent-type extension.
\end{enumerate}


\section*{Addendum (January 9, 2026)}
\begin{itemize}[leftmargin=2em]
    \item Simulation now allows auction covariates: $b_i^{I*} = X_i'\beta + \epsilon_i$ (default remains intercept-only).
    \item Performance metrics compare against $b^{I*}(\bar X)$ (cutoff at mean covariates) when covariates are used.
    \item Sample size $N$ is interpreted as $N_{\text{observed}}$ (conditional on reaching the formal stage).
    \item Planned incomplete-auction handling updates:
    \begin{itemize}
        \item Task A: drop all-reject auctions; include all-accept auctions as one-sided upper bounds.
        \item Task B: drop auctions with zero admitted overall; include type-specific one-sided bounds (including type-specific all-reject lower bounds).
    \end{itemize}
\end{itemize}

\end{document}
